\documentclass{article}
\usepackage{graphicx}
\usepackage[brazilian]{babel}
\usepackage[utf8]{inputenc}
\usepackage[T1]{fontenc}
\usepackage{amsmath}
\usepackage{amssymb}
\setlength{\parindent}{0in}

\begin{document}
	
	\title{Provinha 04 - MAE0119}
	\author{Daniel Yoshio Hotta – 9922700}
	
	\maketitle	
	
		Enviado termo geral.
	
	\textbf {E.a} 
	\\ \\
	\textit {Resposta:} \\
    
    Probabilidade de ter um circuito defeituoso é equivalente a tomarmos um lote e separarmos a proporção das fábricas A, B, C e calcularmos a probabilidade de peças defeituosas de cada fábrica. No caso,  \\
    
    $40\% . 0,01 + 30\% . 0,04 + 30\% . 0, 03$\\
    
    \maketitle	
    
    \textbf {E.b} 
    \\ \\
    \textit {Resposta:} \\
	
    A probabilidade de ser da fábrica C, dado que é um produto defeituoso é dado pela fórmula: \\
    
    $\frac{chance Defeituoso Da Fabrica C}{Total De Chance De Ser Defeituoso (item A)} = \frac{30\% .0,03}{40\% . 0,01 + 30\% . 0,04 + 30\% . 0, 03} $\\
    
	
\end{document}
