\documentclass{article}
\usepackage{graphicx}
\usepackage[brazilian]{babel}
\usepackage[utf8]{inputenc}
\usepackage[T1]{fontenc}
\usepackage{amsmath}
\usepackage{amssymb}
\setlength{\parindent}{0in}
\usepackage{hyperref}

\begin{document}
	
	\title{Classe 12 - MAE0119}
	\author{Daniel Yoshio Hotta – 9922700}
	
	\maketitle	
	
		Enviado termo geral.
	
	\textbf {E.a.} 
	\\ \\
	\textit {Resposta:} \\
    
    Bem, a Esperança será dada pelos valores multiplicados por sua probabilidade, ou seja:
    
    \begin{center}
    	$\sum_{i = 1}^{4} y_i * P(Y = y_i) = (-1.7) * 0.1 + 2.3 * 0.2 + 3.8 * 0.55 + 5.4 * 0.15 = 3.19$\\
    \end{center}
    
    \textbf {E.b.} 
    \\ \\
    \textit {Resposta:} \\
    
    Usarei a fórmula alternativa da $Var(Y) = E(Y^2) - E^2(Y)$.\\
    
    Calculando $E(Y^2)$ de forma quase análoga ao item A (Só elevando ao quadrado os valores de y)
    
    \begin{center}
    	$E(Y^2) = (-1.7)^2 * 0.1 + 2.3^2 * 0.2 + 3.8^2 * 0.55 + 5.4^2 * 0.15 = 13.663$
    \end{center}

    Logo:
    
    \begin{center}
    	$Var(Y) = E(Y^2) - E^2(Y) = 13.663 - 3.19^2 = 3.4869$
    \end{center}
    
    \textbf {E.c.} 
    \\ \\
    \textit {Resposta:} \\
    
    A função de distribuição acumulada de Y é dada por: \\
    
    $F_y(u) = P(Y \leq u), u\in \mathbb{R}$\\
    
    $P(Y \leq u) = $
    $
    \begin{cases}
        0,   &u < -1.7\\
        0.1, & -1.7 \leq u < 2.3\\
        0.3, &  2.3 \leq u < 3.8\\
        0.85,&  3.8 \leq u < 5.4\\
        1,   &  5.4 \leq u
    \end{cases}
	$
	
	    
	\textbf {E.d, e.} 
	\\ \\
	\textit {Resposta:} \\
	
	Código para execução no site \url{https://github.com/HiimHotta/MAE0119/blob/main/Classe12/classe12.py}
	
	
	
\end{document}
