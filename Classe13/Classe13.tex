\documentclass{article}
\usepackage{graphicx}
\usepackage[brazilian]{babel}
\usepackage[utf8]{inputenc}
\usepackage[T1]{fontenc}
\usepackage{amsmath}
\usepackage{amssymb}
\setlength{\parindent}{0in}

\begin{document}
	
	\title{Classe 13 - MAE0119}
	\author{Daniel Yoshio Hotta – 9922700}
	
	\maketitle	
	
		Enviado termo geral.
	
	\textbf {E.a.} 
	\\ \\
	\textit {Resposta:} \\
    
    Temos que calcular a probabilidade que uma família aleatória recicle dada por (FAMÍLIA QUE VÊ O PROGRAMA E RECICLA) OU  (FAMÍLIA QUE NÃO VÊ O PROGRAMA E RECICLA). A conta é dada por:
    
    \begin{center}
    	$\frac{3}{20} * 90\% + \frac{17}{20} * 30\% = 0.135 + 0.255 = 0.390$
    \end{center}

    
    \textbf {E.b.} 
    \\ \\
    \textit {Resposta:} \\
    
    Se eu acertei a questão A e denotarei a probabilidade dela por $P(A)$, então nesses 15 experimentos, queremos contar os que tiveram ao menos 1 sucesso.
    
    \begin{center}
    	$\sum_{i = 1}^{15} P(y = i)$, com $P(y =i) = P(A)^i - (1-P(A))^{15 - i}$
    \end{center}

    No caso, como temos um experimento fechado. Podemos simplificar para $1 - P(y = 0)$ (fiquei em dúvida nessa suposição, mas me pareceu coerente):
    
    \begin{center}
    	$1 - P(y = 0) = 1 - (1 - P(A))^{15} = 1 - (0.61)^{15}$
    \end{center}
    
    \textbf {E.c.} 
    \\ \\
    \textit {Resposta:} \\
    
    A
	
\end{document}
