\documentclass{article}
\usepackage{graphicx}
\usepackage[brazilian]{babel}
\usepackage[utf8]{inputenc}
\usepackage[T1]{fontenc}
\usepackage{amsmath}
\usepackage{amssymb}
\usepackage{hyperref}
\setlength{\parindent}{0in}


\begin{document}
	
	\title{Classe 06 - MAE0119}
	\author{Daniel Yoshio Hotta – 9922700}
	
	\maketitle	
	
	Enviado termo geral.\\
	
	\textbf {E.a} 
	\\ \\
	\textit {Resposta:} \\
    
    Imagino que devam ser menores que as brancas não sei dizer quanto sem cálculos.\\
    
    \maketitle	
    
    \textbf {E.b} 
    \\ \\
    \textit {Resposta:} \\
	
    Simulação dá em torno de 40\% de chance. \\
    
    Link para o código: \url{https://github.com/HiimHotta/MAE0119/blob/main/Classe06/Classe06.py}\\
    
    
    \textbf {E.c} 
    \\ \\
    \textit {Resposta:} \\
    
    Eu lembro de ter aprendido em aula que existe uma árvore de possibilidades condicionadas gerada após essas 9 retiradas, contudo, lembro também que a probabilidade total (de uma altura X de tal árvore) será equivalente à raiz da árvore. No caso, a raiz da árvore tem probabilidade 8/20 = 40%.
    
    \textbf {E.d} 
    \\ \\
    \textit {Resposta:} \\
    
    Eu já tinha aprendido isso anteriormente, contudo, foi muito bom relembrar. Muito legal essa propriedade!
	
\end{document}
