\documentclass{article}
\usepackage{graphicx}
\usepackage[brazilian]{babel}
\usepackage[utf8]{inputenc}
\usepackage[T1]{fontenc}
\usepackage{amsmath}
\usepackage{amssymb}
\usepackage{hyperref}
\setlength{\parindent}{0in}


\begin{document}
	
	\title{Provinha 10 - MAE0119}
	\author{Daniel Yoshio Hotta – 9922700}
	
	\maketitle	
	
	Enviado termo geral.\\
	
	\textbf {E.a} 
	\\ \\
	\textit {Resposta:} \\
    
    Seja $A =$ valor do primeiro dado e $B =$ valor do segundo dado. A variável $X$ seria a quantidade de lançamentos até que distribuição conjunta de $(A, B) = (i, i)$ fosse atingida, enquanto a variável $Y$ seria o número de eventos de $X$ necessários até que os 6 casos fossem alcançados. 
    
    \textbf {E.b} 
    \\ \\
    \textit {Resposta:} \\
    
    Se eu entendi corretamente, seriam meio que a contagem dos fracassos até obtermos um sucesso de algum das variáveis. Uma distribuição geométrica, portanto:
    
    \begin{center}
    	$E(Y) = E(X = (1, 1) + X = (2,2) + ... + X = (6, 6))$
    \end{center}

    Contudo, $X$ a esperança de $(E(X = (i, i)))$, $i = {1, 2..., 6}$ é uma geométrica, com Esperança 1/p, p = 1/36. Logo, $E(Y) = 6 /36$.

\end{document}
