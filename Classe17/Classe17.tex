\documentclass{article}
\usepackage{graphicx}
\usepackage[brazilian]{babel}
\usepackage[utf8]{inputenc}
\usepackage[T1]{fontenc}
\usepackage{amsmath}
\usepackage{amssymb}
\usepackage{hyperref}
\setlength{\parindent}{0in}


\begin{document}
	
	\title{Classe 17 - MAE0119}
	\author{Daniel Yoshio Hotta – 9922700}
	
	\maketitle	
	
	Enviado termo geral.\\
	
	\textbf {E.a} 
	\\ \\
	\textit {Resposta:} \\
    
    Sendo $X_i = N (7.3, 1.7)$, a distribuição de cada toner, $i = {1,2,3,4}$. Queremos calcular a soma deles maior que 30, $P (X_1 + X_2 + X_3 + X_4 > 30) $. E temos um resultado para ele no caso de distribuição normal.
    
    \begin{center}
    	$X_1 + X_2 + X_3 + X_4 aproxima Normal (\mu * 4, \sigma ^ 2 *4) $
    \end{center}

    $P (\frac {30 - 7.3 * 4}{1.7 * 4 })$

    Cálculos...

    \textbf {E.b} 
    \\ \\
    \textit {Resposta:} \\
    
    Pela aula, sabemos que $\bar{X}_4$ é:
    
    \begin{center}
    	$\bar{X}_4 = \frac{X_1 + ... + X_4}{n}$
    \end{center}

    E, para o caso normal, temos:
    
    \begin{center}
    	$\bar{X}_4 aproxima  N (\mu, \frac{\sigma ^2}{n})$
    \end{center}

    Portanto, a $P (\bar{X}_4 \leq 6,3)$ é:
    
    \begin{center}
    	$P (\bar{X}_4 \leq 6,3) = P (Z \leq \frac{6,3 - 7,3}{1,7}) = P (Z \leq -0.58824)$
    \end{center}

    Aproximando um pouco, temos: $P(Z \leq -0.59) = 0.5 - 0.22240 = 0.27776$

\end{document}
