\documentclass{article}
\usepackage{graphicx}
\usepackage[brazilian]{babel}
\usepackage[utf8]{inputenc}
\usepackage[T1]{fontenc}
\usepackage{amsmath}
\usepackage{amssymb}
\usepackage{hyperref}
\setlength{\parindent}{0in}


\begin{document}
	
	\title{Classe 18 - MAE0119}
	\author{Daniel Yoshio Hotta – 9922700}
	
	\maketitle	
	
	Enviado termo geral.\\
	
	\textbf {E.a} 
	\\ \\
	\textit {Resposta:} \\
    
    Temos que calcular a área central da distribuição Normal, ou seja:\\
    
    \begin{center}
    	$P (135 < Y < 165) = 1 - P (135 < Y) - P (Y < 165) = $\\
    	$P (135 < Y < 165) = 1 - P (\frac{135 - 150}{23} < Z) - P (Z < \frac{165-150}{23}) = $\\
    	$P (135 < Y < 165) = 1 - P (-0.6522 < Z) - P (Z < 0.6522) = $\\
    	$P (135 < Y < 165) = 1 - 0.24215 - 0.24215 = 0.5157$
    \end{center}
    
    \textbf {E.b} 
    \\ \\
    \textit {Resposta:} \\	
        
    Sabemos que o peso médio de uma amostra aleatória com distribuição Normal ($Y_{16}$)aproxima da seguinte forma $Normal (\mu, \frac{\sigma}{16})$, logo:\\
    
    \begin{center}
    	$P(135 < Y_{16} < 165) = 1 - P (135 < Y_{16}) - P (Y_{16} < 165) = $\\
    	$P(135 < Y_{16} < 165) = 1 - P (\frac{135 - 150}{\sqrt{23^2 / 16}}< Z) - P (Z < \frac{165 - 150}{\sqrt{23^2 / 16}})$\\
    	$P(135 < Y_{16} < 165) = 1 - P (-2.6087 < Z) - P(Z < 2.6087)$\\
    	$P(135 < Y_{16} < 165) = 1 - 49547 - 49547 = 0.00906$
    \end{center}

    \textbf {E.c} 
    \\ \\
    \textit {Resposta:} \\
    
    Nesse caso, queremos descobrir o valor de $n$, contudo, precisamos calcular $P (Y_n < 165) = 0.025$ primeiro (Normal é equidistante nesse caso) \\
    
    \begin{center}
    	$P(Y_n < 165) = P (Z < \frac{165 - 150}{\sqrt{23^2 / n}}) = 0.025$
    	$Z = 0.67$
    \end{center}
    
    Logo, temos:
    
    \begin{center}
    	$\frac{160 - 150}{\sqrt{23^2 / n}} = 0.67$
    	$15 = 0.67 * \sqrt{23^2 / n}$
    \end{center}

\end{document}
