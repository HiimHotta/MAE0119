\documentclass{article}
\usepackage{graphicx}
\usepackage[brazilian]{babel}
\usepackage[utf8]{inputenc}
\usepackage[T1]{fontenc}
\usepackage{amsmath}
\usepackage{amssymb}
\setlength{\parindent}{0in}

\begin{document}
	
	\title{Classe 10 - MAE0119}
	\author{Daniel Yoshio Hotta – 9922700}
	
	\maketitle	
	
		Enviado termo geral.
	
	\textbf {E} 
	\\ \\
	\textit {Resposta:} \\
    
    Iremos fazer os 6 primeiros para ter uma noção da fórmula geral da função.
    
    \begin{center}
    	$F(Y = 10) = 0,7$\\
    	$F(Y = 20) = 0,7^2$\\
    	$F(Y = 30) = 0,7^3$\\
    	$F(Y = 35) = 0,7^4$\\
    	$F(Y = 40) = 0,7^5$\\
    	$F(Y = 45) = 0,7^6$\\
    	....\\
    \end{center}
    
    Ora, então temos que a variável $Y$ pode assumir $10$, $20$, $30$ nos 3 primeiros experimentos e, a partir daí, ela assume todos os valores múltiplos de 5. Ou seja, acima de 3 experimentos teremos que $Y$ assume os valores da forma $Y_j = 30 + 5j$, $j = 1, 2, ....$. Logo, a fórmula geral seria mais ou menos(se não tiver erro de notação):
    
    \begin{center}
    	$p(y) = P(Y = y) = 0,7^i $, com i sendo o índice do vetor $yi = {10, 20, 30, 35, 40, ...}$
    \end{center}
    
    
    
	
\end{document}
