\documentclass{article}
\usepackage{graphicx}
\usepackage[brazilian]{babel}
\usepackage[utf8]{inputenc}
\usepackage[T1]{fontenc}
\usepackage{amsmath}
\usepackage{amssymb}
\usepackage{hyperref}
\setlength{\parindent}{0in}


\begin{document}
	
	\title{Provinha 12 - MAE0119}
	\author{Daniel Yoshio Hotta – 9922700}
	
	\maketitle	
	
	Enviado termo geral.\\
	
	\textbf {E.a} 
	\\ \\
	\textit {Resposta:} \\
    
    Sendo X a variável aleatória que com distribuição Normal $N(\mu = 206.4, \sigma^2 = 45g^2)$ que representa o peso dos pacotes.\\
    
    Temos que calcular a probabilidade de $\bar {X} = X_1 + X_2 + X_3 + X_4 + X_5$, onde $X_i, i = {1, ..., 5}$ sao variaveis com a distribuicao acima. Sabemos pelo visto em aula que isso na verdade eh aproximado por:\\
    
    \begin{center}
    	$X_5$ aproxima $Normal (\mu, \sigma ^2 / 5) $
    \end{center}

    Logo, calculando a probabilidade dos eventos acontecerem é:
    
    \begin{center}
    	$P(X \leq 197$ ou $X \geq 210)$ $ = 1 - P(197 \leq X \leq 210)$\\
    	$P(X \leq 197$ ou $X \geq 210)$ $ = 1 - P (\frac {197 - 206.4}{\sqrt{45^2/5}} \leq Z \leq \frac {210 - 206.4}{\sqrt{45^2/5}})$\\    	
    	$P(X \leq 197$ ou $X \geq 210)$ $ = P (-0.4671 \leq Z \leq 0.1789)$\\
    	$P(X \leq 197$ ou $X \geq 210)$ $ = 1 - 0.18082 - 0.07142 = 0.74776$
    \end{center}
    
    \textbf {E.b} 
    \\ \\
    \textit {Resposta:} \\	
        
    Agora, queremos saber o evento em que o X se manteve dentro da produção, ou seja, $P (197 < X < 210)$, contudo, com $\mu = 200$:\\
    
    \begin{center}
    	$P(197 \leq X \leq 210) = P (\frac {197 - 200}{\sqrt{45^2/5}} \leq Z \leq \frac {210 - 200}{\sqrt{45^2/5}})$\\
    	$P(197 \leq X \leq 210) = P (-0.1491 \leq Z \leq 0.4969)$\\
    	$P(197 \leq X \leq 210) = 0.05962 + 0.19146 = 0.25108$
    \end{center}



\end{document}
