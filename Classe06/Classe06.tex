\documentclass{article}
\usepackage{graphicx}
\usepackage[brazilian]{babel}
\usepackage[utf8]{inputenc}
\usepackage[T1]{fontenc}
\usepackage{amsmath}
\usepackage{amssymb}
\setlength{\parindent}{0in}

\begin{document}
	
	\title{Classe 06 - MAE0119}
	\author{Daniel Yoshio Hotta – 9922700}
	
	\maketitle	
	
		Enviado termo geral.
	
	\textbf {E.a} 
	\\ \\
	\textit {Resposta:} \\
    
    Se eles são disjuntos, então temos que a probabilidade de gostar de gatos é 1/3. Uma vez que não tem como gostar dos dois ao mesmo tempo e é justamente isso que queremos. \\
    
    \maketitle	
    
    \textbf {E.b} 
    \\ \\
    \textit {Resposta:} \\
	
    Se todos que gostam de gatos também gostam de cachorros, isso implica que não podemos ter pessoas que gostam de gatos E não gostam de cachorros. Ou seja, $P(G\cap \neg C) = 0$. \\
    
    \textbf {E.c} 
    \\ \\
    \textit {Resposta:} \\
    
    Temos que ter a probabilidade total de gostar de gatos igual a 1/3, portanto, o item c fornece o complemento do que queremos, ou seja, o valor final é 1/3 - 1/8 = 5/24.
	
\end{document}
