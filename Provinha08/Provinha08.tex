\documentclass{article}
\usepackage{graphicx}
\usepackage[brazilian]{babel}
\usepackage[utf8]{inputenc}
\usepackage[T1]{fontenc}
\usepackage{amsmath}
\usepackage{amssymb}
\usepackage{hyperref}
\setlength{\parindent}{0in}


\begin{document}
	
	\title{Provinha 08 - MAE0119}
	\author{Daniel Yoshio Hotta – 9922700}
	
	\maketitle	
	
	Enviado termo geral.\\
	
	\textbf {E.a} 
	\\ \\
	\textit {Resposta:} \\
    
    Média: 90 min, DP = 20 min. $Y ~ Normal (90, 20)$\\
    
    Nesse item, queremos a $P(Y < 80)$. Portanto, temos que calcular o $z$:
    
    \begin{center}
    	$z = \frac{x - \mu}{\sigma} = \frac{80 - 90}{20} = -0.5$
    \end{center}
    
    Portanto, temos que $P (z_c < -0.5) = 0.5 - 0.19146 = 0.30854$, para uma amostra de 65 pessoas, temos: 
    
    \begin{center}
    	$65 * P (z_c < -0.5) = 65 * 0.30854 = 20.0551$ pessoas
    \end{center}
    
    \textbf {E.b} 
    \\ \\
    \textit {Resposta:} \\	
        
    Nesse caso, temos que fazer o processo inverso para descobrir o $x$, sabemos que $P(z_c \leq z) = 0.05$, logo, pela tabela, $z = -1.64$. Portanto,\\
    
    \begin{center}
    	$z = \frac{x - \mu}{\sigma}$\\
    	$-1.64 = \frac{x - 90}{20} = $\\
    	$x = 57.2 min$
    \end{center}

\end{document}
