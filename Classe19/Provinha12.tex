\documentclass{article}
\usepackage{graphicx}
\usepackage[brazilian]{babel}
\usepackage[utf8]{inputenc}
\usepackage[T1]{fontenc}
\usepackage{amsmath}
\usepackage{amssymb}
\usepackage{hyperref}
\setlength{\parindent}{0in}


\begin{document}
	
	\title{Classe 19 - MAE0119}
	\author{Daniel Yoshio Hotta – 9922700}
	
	\maketitle	
	
	Enviado termo geral.\\
	
	\textbf {E.a} 
	\\ \\
	\textit {Resposta:} \\
    
    Queremos encontrar $n$ tal que $P (|\hat{p} - p| \leq 4\%) = 90\%$. Sabemos que cada paciente tem distribuição Bernoulli(p) e, em particular, podemos usar o TLC para $n$ grande tal que:
    
    \begin{center}
    	$\epsilon = z_\gamma . \sqrt{\frac{p(1 - p)}{n}}$\\
    	$0.4 = 1.64 . \sqrt{\frac{p(1 - p)}{n}}$\\
    	$n \leq (\frac{1.64 * \sqrt{p (1-p)}}{0.4})^2$\\
    	$n \leq (\frac{1.64 * \sqrt{1/4}}{0.4})^2$
    	$n \leq (4.1/2)^2
    \end{center}
    
    
    \textbf {E.b} 
    \\ \\
    \textit {Resposta:} \\	
        
    


\end{document}
