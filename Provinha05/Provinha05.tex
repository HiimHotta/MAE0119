\documentclass{article}
\usepackage{graphicx}
\usepackage[brazilian]{babel}
\usepackage[utf8]{inputenc}
\usepackage[T1]{fontenc}
\usepackage{amsmath}
\usepackage{amssymb}
\setlength{\parindent}{0in}

\begin{document}
	
	\title{Provinha 05 - MAE0119}
	\author{Daniel Yoshio Hotta – 9922700}
	
	\maketitle	
	
		Enviado termo geral.
	
	\textbf {E.a.} 
	\\ \\
	\textit {Resposta:} \\
    
    Segue a função de probabilidade calculada ao pegarmos o total e subtrairmos o anterior (já que foi fornecida a Acumulada).\\
    
    $	
    	F(u) = 
    	\begin{cases}
    		0  , & \mbox{se } u < -2\\
    		0.1, & \mbox{se } -2 \leq u \leq -1\\
    		0.5, & \mbox{se } -1 \leq u \leq 0\\
    		0.3, & \mbox{se } 0 \leq u \leq 4\\
    		0.1, & \mbox{se } u \geq 4\\
    	\end{cases}
    $

    
    \textbf {E.b.} 
    \\ \\
    \textit {Resposta:} \\
    
    A esperança de uma v.a. Y pode ser dada pelos valores que pode assumir (no caso, a média desse intervalo) multiplicada pela sua probabilidade, no caso:
    
    $ 0.1 * -1.5 + 0.5 * -0.5 + 0.3 * 2 + 0.1 * 4$
    
    \textbf {E.c.} 
    \\ \\
    \textit {Resposta:} \\
    
    A
	
\end{document}
