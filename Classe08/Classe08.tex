\documentclass{article}
\usepackage{graphicx}
\usepackage[brazilian]{babel}
\usepackage[utf8]{inputenc}
\usepackage[T1]{fontenc}
\usepackage{amsmath}
\usepackage{amssymb}
\usepackage{hyperref}
\setlength{\parindent}{0in}


\begin{document}
	
	\title{Classe 08 - MAE0119}
	\author{Daniel Yoshio Hotta – 9922700}
	
	\maketitle	
	
	Enviado termo geral.\\
	
	\textbf {E.a} 
	\\ \\
	\textit {Resposta:} \\
    
    Pra sabermos a probabilidade de ser do setor técnico, temos que ter: $\frac{quant. tecnico}{total} = \frac{800 + 2500 + 2200}{15800} = \frac{5500}{15800}$\\
    
    \maketitle	
    
    \textbf {E.b} 
    \\ \\
    \textit {Resposta:} \\
	
    Similar ao anterior, a probabilidade é dada pela quantidade do evento em particular dividido pelo total de possibilidades. Ou seja, $\frac{1800+800}{15800} = \frac{2600}{15800}$ \\
    
    \textbf {E.c} 
    \\ \\
    \textit {Resposta:} \\
    
    Se sabemos que é do setor técnico, logo a quantidade de possibilidades totais é $600 + 200 + 1400 + 1100 + 1400 + 800$ = $800 + 2500 + 2200 = 5500$.\\
    
    E a quantidade de funcionários "até 25" dado que são somente do setor técnico é: $\frac{800}{5500}$.\\
    
    \textbf {E.d} 
    \\ \\
    \textit {Resposta:} \\
    
    A quantidade de pessoas do sexo feminino é: $900 + 2200 + 1800 + 200 + 1100 + 800 = 7000$.\\
    
    Agora, a probabilidade de ser do setor técnico dado que são mulheres é: $\frac{2100}{7000}$.\\
	
\end{document}
